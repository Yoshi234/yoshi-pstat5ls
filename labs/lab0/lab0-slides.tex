% Options for packages loaded elsewhere
\PassOptionsToPackage{unicode}{hyperref}
\PassOptionsToPackage{hyphens}{url}
\documentclass[
  ignorenonframetext,
]{beamer}
\newif\ifbibliography
\usepackage{pgfpages}
\setbeamertemplate{caption}[numbered]
\setbeamertemplate{caption label separator}{: }
\setbeamercolor{caption name}{fg=normal text.fg}
\beamertemplatenavigationsymbolsempty
% remove section numbering
\setbeamertemplate{part page}{
  \centering
  \begin{beamercolorbox}[sep=16pt,center]{part title}
    \usebeamerfont{part title}\insertpart\par
  \end{beamercolorbox}
}
\setbeamertemplate{section page}{
  \centering
  \begin{beamercolorbox}[sep=12pt,center]{section title}
    \usebeamerfont{section title}\insertsection\par
  \end{beamercolorbox}
}
\setbeamertemplate{subsection page}{
  \centering
  \begin{beamercolorbox}[sep=8pt,center]{subsection title}
    \usebeamerfont{subsection title}\insertsubsection\par
  \end{beamercolorbox}
}
% Prevent slide breaks in the middle of a paragraph
\widowpenalties 1 10000
\raggedbottom
\AtBeginPart{
  \frame{\partpage}
}
\AtBeginSection{
  \ifbibliography
  \else
    \frame{\sectionpage}
  \fi
}
\AtBeginSubsection{
  \frame{\subsectionpage}
}
\usepackage{iftex}
\ifPDFTeX
  \usepackage[T1]{fontenc}
  \usepackage[utf8]{inputenc}
  \usepackage{textcomp} % provide euro and other symbols
\else % if luatex or xetex
  \usepackage{unicode-math} % this also loads fontspec
  \defaultfontfeatures{Scale=MatchLowercase}
  \defaultfontfeatures[\rmfamily]{Ligatures=TeX,Scale=1}
\fi
\usepackage{lmodern}
\usetheme[]{AnnArbor}
\ifPDFTeX\else
  % xetex/luatex font selection
\fi
% Use upquote if available, for straight quotes in verbatim environments
\IfFileExists{upquote.sty}{\usepackage{upquote}}{}
\IfFileExists{microtype.sty}{% use microtype if available
  \usepackage[]{microtype}
  \UseMicrotypeSet[protrusion]{basicmath} % disable protrusion for tt fonts
}{}
\makeatletter
\@ifundefined{KOMAClassName}{% if non-KOMA class
  \IfFileExists{parskip.sty}{%
    \usepackage{parskip}
  }{% else
    \setlength{\parindent}{0pt}
    \setlength{\parskip}{6pt plus 2pt minus 1pt}}
}{% if KOMA class
  \KOMAoptions{parskip=half}}
\makeatother
\usepackage{color}
\usepackage{fancyvrb}
\newcommand{\VerbBar}{|}
\newcommand{\VERB}{\Verb[commandchars=\\\{\}]}
\DefineVerbatimEnvironment{Highlighting}{Verbatim}{commandchars=\\\{\}}
% Add ',fontsize=\small' for more characters per line
\usepackage{framed}
\definecolor{shadecolor}{RGB}{248,248,248}
\newenvironment{Shaded}{\begin{snugshade}}{\end{snugshade}}
\newcommand{\AlertTok}[1]{\textcolor[rgb]{0.94,0.16,0.16}{#1}}
\newcommand{\AnnotationTok}[1]{\textcolor[rgb]{0.56,0.35,0.01}{\textbf{\textit{#1}}}}
\newcommand{\AttributeTok}[1]{\textcolor[rgb]{0.13,0.29,0.53}{#1}}
\newcommand{\BaseNTok}[1]{\textcolor[rgb]{0.00,0.00,0.81}{#1}}
\newcommand{\BuiltInTok}[1]{#1}
\newcommand{\CharTok}[1]{\textcolor[rgb]{0.31,0.60,0.02}{#1}}
\newcommand{\CommentTok}[1]{\textcolor[rgb]{0.56,0.35,0.01}{\textit{#1}}}
\newcommand{\CommentVarTok}[1]{\textcolor[rgb]{0.56,0.35,0.01}{\textbf{\textit{#1}}}}
\newcommand{\ConstantTok}[1]{\textcolor[rgb]{0.56,0.35,0.01}{#1}}
\newcommand{\ControlFlowTok}[1]{\textcolor[rgb]{0.13,0.29,0.53}{\textbf{#1}}}
\newcommand{\DataTypeTok}[1]{\textcolor[rgb]{0.13,0.29,0.53}{#1}}
\newcommand{\DecValTok}[1]{\textcolor[rgb]{0.00,0.00,0.81}{#1}}
\newcommand{\DocumentationTok}[1]{\textcolor[rgb]{0.56,0.35,0.01}{\textbf{\textit{#1}}}}
\newcommand{\ErrorTok}[1]{\textcolor[rgb]{0.64,0.00,0.00}{\textbf{#1}}}
\newcommand{\ExtensionTok}[1]{#1}
\newcommand{\FloatTok}[1]{\textcolor[rgb]{0.00,0.00,0.81}{#1}}
\newcommand{\FunctionTok}[1]{\textcolor[rgb]{0.13,0.29,0.53}{\textbf{#1}}}
\newcommand{\ImportTok}[1]{#1}
\newcommand{\InformationTok}[1]{\textcolor[rgb]{0.56,0.35,0.01}{\textbf{\textit{#1}}}}
\newcommand{\KeywordTok}[1]{\textcolor[rgb]{0.13,0.29,0.53}{\textbf{#1}}}
\newcommand{\NormalTok}[1]{#1}
\newcommand{\OperatorTok}[1]{\textcolor[rgb]{0.81,0.36,0.00}{\textbf{#1}}}
\newcommand{\OtherTok}[1]{\textcolor[rgb]{0.56,0.35,0.01}{#1}}
\newcommand{\PreprocessorTok}[1]{\textcolor[rgb]{0.56,0.35,0.01}{\textit{#1}}}
\newcommand{\RegionMarkerTok}[1]{#1}
\newcommand{\SpecialCharTok}[1]{\textcolor[rgb]{0.81,0.36,0.00}{\textbf{#1}}}
\newcommand{\SpecialStringTok}[1]{\textcolor[rgb]{0.31,0.60,0.02}{#1}}
\newcommand{\StringTok}[1]{\textcolor[rgb]{0.31,0.60,0.02}{#1}}
\newcommand{\VariableTok}[1]{\textcolor[rgb]{0.00,0.00,0.00}{#1}}
\newcommand{\VerbatimStringTok}[1]{\textcolor[rgb]{0.31,0.60,0.02}{#1}}
\newcommand{\WarningTok}[1]{\textcolor[rgb]{0.56,0.35,0.01}{\textbf{\textit{#1}}}}
\usepackage{graphicx}
\makeatletter
\newsavebox\pandoc@box
\newcommand*\pandocbounded[1]{% scales image to fit in text height/width
  \sbox\pandoc@box{#1}%
  \Gscale@div\@tempa{\textheight}{\dimexpr\ht\pandoc@box+\dp\pandoc@box\relax}%
  \Gscale@div\@tempb{\linewidth}{\wd\pandoc@box}%
  \ifdim\@tempb\p@<\@tempa\p@\let\@tempa\@tempb\fi% select the smaller of both
  \ifdim\@tempa\p@<\p@\scalebox{\@tempa}{\usebox\pandoc@box}%
  \else\usebox{\pandoc@box}%
  \fi%
}
% Set default figure placement to htbp
\def\fps@figure{htbp}
\makeatother
\setlength{\emergencystretch}{3em} % prevent overfull lines
\providecommand{\tightlist}{%
  \setlength{\itemsep}{0pt}\setlength{\parskip}{0pt}}
\usepackage{bookmark}
\IfFileExists{xurl.sty}{\usepackage{xurl}}{} % add URL line breaks if available
\urlstyle{same}
\hypersetup{
  pdftitle={PSTAT 5LS Lab 0},
  pdfauthor={Joshua Lee},
  hidelinks,
  pdfcreator={LaTeX via pandoc}}

\title{PSTAT 5LS Lab 0}
\author{Joshua Lee}
\date{Fall 2025}

\begin{document}
\frame{\titlepage}

\section{Introduction to PSTAT 5LS}\label{introduction-to-pstat-5ls}

\begin{frame}{Introduction}
\phantomsection\label{introduction}
Hi everyone, I'm Josh! - PhD in the PSTAT department - Fan of piano /
classical music - F1 Racing enthusiast

Office Hours: - \textbf{Building 434, Room 109: 4pm to 5:30pm on
FRIDAYS}

Contact: - \href{mailto:jlee246@ucsb.edu}{\nolinkurl{jlee246@ucsb.edu}}
\end{frame}

\begin{frame}{Section Logistics}
\phantomsection\label{section-logistics}
\begin{itemize}
\tightlist
\item
  Sections meet twice a week (Tuesdays/Thursdays 11am - 11:50am)
\item
  Section attendance is worth 10\% of your overall grade in PSTAT 5LS
\item
  We understand that sometimes you'll need to miss section, because
  ``life happens''. Our course policy will be to drop your lowest 3
  attendance scores (absences included)
\item
  Use these 3 drops wisely! Save them for a rainy day!
\end{itemize}
\end{frame}

\begin{frame}{Homework Logistics}
\phantomsection\label{homework-logistics}
\begin{itemize}
\tightlist
\item
  Each week there will be a homework assignment through Gradescope
\item
  Homework is open for 7 days and is due before 11:59 p.m. Pacific on
  Thursdays
\item
  Late work is not accepted and extensions are not permitted because the
  solutions are posted at the deadline
\item
  Your lowest 1 homework score will be dropped
\item
  Use this 1 HW drop wisely! Don't just blow off homework one week and
  then wish you had that drop later in the quarter!
\end{itemize}
\end{frame}

\section{Learning Objectives}\label{learning-objectives}

\begin{frame}{R Learning Objectives}
\phantomsection\label{r-learning-objectives}
\begin{enumerate}
\tightlist
\item
  Learn the difference between R, RStudio, and R Markdown
\item
  Become familiar with the RStudio interface
\item
  Understand key components of an R Markdown document
\item
  Learn how to use R to do basic calculations
\end{enumerate}
\end{frame}

\begin{frame}{Statistical Learning Objectives}
\phantomsection\label{statistical-learning-objectives}
\begin{enumerate}
\tightlist
\item
  Understand that technology can be helpful in statistics
\end{enumerate}
\end{frame}

\section{Lab Tutorial}\label{lab-tutorial}

\begin{frame}{Getting Started: What is R?}
\phantomsection\label{getting-started-what-is-r}
In Statistics, we often use computers to analyze data. There are a lot
of ``statistical computing environments'' that can help you do
statistical analyses. One of the most popular (and powerful) is called
R. R is a programming language that is designed for manipulating data,
doing calculations, and making graphical displays. R works by writing
\textbf{R code}.

That might sound scary, but \emph{don't worry}: this is not a
programming class. Over the course of the quarter, you'll learn how to
edit and write some basic R code to help you analyze data to answer
research questions. Our goal with PSTAT 5LS labs is to help you learn
the basics of R and R coding, but through the lens of answering
statistical questions.
\end{frame}

\begin{frame}{What are all these ``R'' terms?}
\phantomsection\label{what-are-all-these-r-terms}
There are a lot of ``R'' words floating around. What's going on?

\begin{itemize}
\tightlist
\item
  \emph{R} is a ``statistical computing environment'' that's designed
  for manipulating data, generating plots, and performing analyses. It's
  also a programming language. You'll be \emph{using R} this quarter.
\item
  \emph{RStudio} is an ``integrated development environment (IDE)'' for
  R (you'll never have to hear the term IDE again in this class).
  Basically, RStudio is a pretty interface that makes working with R
  easier. You use R inside of RStudio. If R is ice cream, RStudio is the
  cone or cup.
\item
  \emph{R Markdown} is a way to write pretty analysis reports that
  combines R code, R output (plots, analysis results, etc.) and text in
  one document. This lab document is an R Markdown report!
\end{itemize}
\end{frame}

\begin{frame}{JupyterHub}
\phantomsection\label{jupyterhub}
We will be accessing R and RStudio through JupyterHub. JupyterHub is a
web-based multiuser environment that will allow us to share with you the
files you need for lab. LSIT has provided us with access to JupyterHub.

Each time we begin a lab or when you work on a data analysis question
for homework,navigate to \url{http://bit.ly/47PSVM6}. This link will
import whatever we are working on for that day. If you just want to open
a session and not import the lab or a blank homework file into your
directoroy, use this link here. \url{https://pstat5ls.lsit.ucsb.edu/}.

This will open your ``instance'' of JupyterHub and copy all necessary
files for your PSTAT 5LS work that week.
\end{frame}

\begin{frame}{Initial View of RStudio}
\phantomsection\label{initial-view-of-rstudio}
Click on the `pstat5ls-fall25' folder.

\begin{center}
\includegraphics[]{assets/images/RStudio-initial.png}
\end{center}
\end{frame}

\begin{frame}{Accessing the Lab}
\phantomsection\label{accessing-the-lab}
After you have clicked on the `pstat-5ls-f25-student' folder, click on
the folder for this lab (lab0-intro-to-R) to open it:

\begin{center}
\includegraphics[width=10cm]{assets/images/lab0-intro-to-R.png}
\end{center}
\end{frame}

\begin{frame}{RStudio Project Screen}
\phantomsection\label{rstudio-project-screen}
\pandocbounded{\includegraphics[keepaspectratio]{assets/images/r_homescreen.png}}
\end{frame}

\begin{frame}{Working with R in JupyterHub}
\phantomsection\label{working-with-r-in-jupyterhub}
When you navigate to \url{http://bit.ly/47PSVM6}, you are telling
JupyterHub to bring in all of the files that have prepared for PSTAT
5LS. This means that any files you have worked on will be overwritten.

To avoid your work being overwritten, \textbf{rename} both the notes and
report files. For Lab 0, you want to rename lab0-notes.Rmd. You could
add your name to the file name (e.g., lab0-notes-brianw.Rmd).
\end{frame}

\begin{frame}{Renaming Files}
\phantomsection\label{renaming-files}
To rename a file, check the box next to the file you want to rename.
Click ``Rename'' in the files pane, rename the file, then click OK.

You may also save your work in a separate folder, and there will never
be any danger of it being overwritten.

\vspace{0.25cm}
\begin{center}
\includegraphics[width=8cm]{assets/images/beforeRename.png}
\end{center}
\end{frame}

\begin{frame}{Renaming Files}
\phantomsection\label{renaming-files-1}
Your file will then appear with its new name.

\vspace{0.25cm}
\begin{center}
\includegraphics[width=8cm]{assets/images/afterRename.png}
\end{center}
\end{frame}

\begin{frame}[fragile]{R Markdown}
\phantomsection\label{r-markdown}
R Markdown lets you combine text, R code, and plots in one pretty,
reproducible report. If you're curious about this, you can find more
details on using R Markdown at \url{http://rmarkdown.rstudio.com}.

R Markdown runs code contained in ``chunks''. A chunk looks like this:

\pandocbounded{\includegraphics[keepaspectratio]{assets/images/chunk.png}}

\begin{Shaded}
\begin{Highlighting}[]
\FunctionTok{print}\NormalTok{(}\StringTok{"Hello world!"}\NormalTok{)}
\end{Highlighting}
\end{Shaded}

\begin{verbatim}
## [1] "Hello world!"
\end{verbatim}
\end{frame}

\begin{frame}[fragile]{R Markdown Chunk}
\phantomsection\label{r-markdown-chunk}
\pandocbounded{\includegraphics[keepaspectratio]{assets/images/chunk.png}}

Notice that the code, \texttt{print("Hello\ world!")} is contained
between three backticks
(\texttt{\textasciigrave{}\textasciigrave{}\textasciigrave{}}, below the
esc key on a US English keyboard; the same key as the tilde) and some
stuff in curly braces \{\}. This is how R Markdown knows where your
chunks start and stop. We will almost always provide pre-made chunks for
you to use.
\end{frame}

\begin{frame}[fragile]{Running a Chunk}
\phantomsection\label{running-a-chunk}
You can just run a single chunk by clicking the green ``play'' button in
the upper right corner of the chunk. It's usually a good idea to click
the ``Run All Chunks Above'' button immediately to the left of the play
button first. So you should click Run All Chunks Above, then the play
button.

\pandocbounded{\includegraphics[keepaspectratio]{assets/images/run_chunk.png}}

If you can't find the play button, just highlight the code you want to
run and click the Run button in the top right corner of the editor pane.

Let's try this code together in the \texttt{tryIt1} chunk in your notes
document. Note that the code is already entered for you in this first
code chunk.
\end{frame}

\begin{frame}{Knitting a Document}
\phantomsection\label{knitting-a-document}
When you click the \textbf{Knit} button in RStudio, a document will be
generated that includes both content as well as the output of any
embedded R code chunks within the document.

\pandocbounded{\includegraphics[keepaspectratio]{assets/images/knit-button.png}}
\end{frame}

\begin{frame}[fragile]{Tips for R Markdown}
\phantomsection\label{tips-for-r-markdown}
\begin{enumerate}
\item
  \textbf{Knit and knit often}: Frequently knitting your document will
  help you make sure that all your code works and that the document
  looks the way you want.
\item
  \textbf{Don't be afraid to experiment}: Nobody gets things exactly
  right the first time, and we all forget how things work sometimes.
  Keep trying, and you'll eventually get what you want!
\item
  \textbf{Formatting}: You can make text \textbf{bold} by surrounding it
  with two asterisks (\texttt{**}) and \emph{italic} by surrounding it
  with one asterisk (\texttt{*}). You can make text
  \texttt{look\ like\ code} by surrounding it with single backticks.
\end{enumerate}
\end{frame}

\begin{frame}[fragile]{Using R as a Calculator}
\phantomsection\label{using-r-as-a-calculator}
At it's most basic, R is a fancy calculator.

Let's try this example together in the \texttt{tryIt2} chunk in your lab
notes!

Remember, you can run the code in this chunk by clicking the green
``play'' button in the upper right corner of the chunk.

\begin{Shaded}
\begin{Highlighting}[]
\DecValTok{5} \SpecialCharTok{*} \DecValTok{7}
\end{Highlighting}
\end{Shaded}

\begin{verbatim}
## [1] 35
\end{verbatim}

When you run the chunk, you'll see a \texttt{{[}1{]}} before the output
of \texttt{35}. \emph{Just ignore this. The result is \texttt{35}.}

All of the symbols you think you'd use to do math work: \texttt{+} for
addition, \texttt{-} for subtraction, \texttt{*} for multiplication,
\texttt{/} for division, \texttt{\^{}} for exponentiation.
\end{frame}

\begin{frame}[fragile]{R Function Syntax}
\phantomsection\label{r-function-syntax}
We'll be using R primarily through the use of \emph{functions}. A
function generally looks something like this:

\begin{Shaded}
\begin{Highlighting}[]
\FunctionTok{function.name}\NormalTok{(argument1, argument2, ...)}
\end{Highlighting}
\end{Shaded}

Notice that we have the name of the function, followed immediately by an
open parenthesis \textcolor{blue}{`(`} without a space, then a sequence
of \emph{arguments} to the function, separated by commas
\textcolor{blue}{`,`}, then a close parenthesis \textcolor{blue}{`)`}.
\end{frame}

\begin{frame}[fragile]{R ``Assignment'' Syntax}
\phantomsection\label{r-assignment-syntax}
Often, we want R to remember the results of a calculation so that we can
use it later. We can give the result of some code a name by
\textbf{assigning} it to something.

\begin{Shaded}
\begin{Highlighting}[]
\NormalTok{x }\OtherTok{\textless{}{-}} \DecValTok{36} \SpecialCharTok{/} \DecValTok{6}
\end{Highlighting}
\end{Shaded}

We read that code as ``x \emph{gets} 36 / 6''. The arrow is made using
the less-than symbol (\texttt{\textless{}}, shift + comma on a US
English keyboard) and a hyphen.

Fill out the \texttt{tryIt3} code chunk in your notes to assign 36
divided by 6 to the variable \texttt{x}.
\end{frame}

\begin{frame}[fragile]{R ``Assignment'' Syntax}
\phantomsection\label{r-assignment-syntax-1}
Now, we've stored the result as \texttt{x}, and R will remember that
\texttt{x} is 6. You can see in the environment pane in RStudio (top
right) that there's now a ``value'' called \texttt{x} and it's 6. You
can also access the value of \texttt{x} by typing \texttt{x} into R.
Check it out:

\begin{Shaded}
\begin{Highlighting}[]
\NormalTok{x}
\end{Highlighting}
\end{Shaded}

\begin{verbatim}
## [1] 6
\end{verbatim}

Use the \texttt{tryIt4} code chunk in your notes to see the value that R
stored for the variable \texttt{x}.
\end{frame}

\begin{frame}[fragile]{R is Case-Sensitive}
\phantomsection\label{r-is-case-sensitive}
R is ``case-sensitive'', which means that upper-case letters are
\emph{different from} lower-case letters. Notice what happens when we
ask R for the value of \texttt{X}:

\begin{Shaded}
\begin{Highlighting}[]
\NormalTok{X}
\end{Highlighting}
\end{Shaded}

\begin{verbatim}
## Error in eval(expr, envir, enclos): object 'X' not found
\end{verbatim}

Use the \texttt{tryIt5} code chunk in your notes to see what happens if
you type in a capital X.

When giving things names in R, you can only use a combination of
letters, numbers, periods, and underscores, and the \textbf{names have
to start with a letter or a period}. People tend to use underscores or
periods instead of spaces.
\end{frame}

\begin{frame}[fragile]{Common Errors}
\phantomsection\label{common-errors}
Watch what happens when you try to assign something to a ``bad'' name:

\small

\begin{Shaded}
\begin{Highlighting}[]
\NormalTok{tik tok }\OtherTok{\textless{}{-}} \DecValTok{12}
\end{Highlighting}
\end{Shaded}

\begin{verbatim}
## Error: <text>:1:5: unexpected symbol
## 1: tik tok
##         ^
\end{verbatim}

\begin{Shaded}
\begin{Highlighting}[]
\DecValTok{4}\NormalTok{eva }\OtherTok{\textless{}{-}} \DecValTok{4} \SpecialCharTok{*} \DecValTok{2}
\end{Highlighting}
\end{Shaded}

\begin{verbatim}
## Error: <text>:1:1: unexpected input
## 1: 4ev
##     ^
\end{verbatim}

\begin{Shaded}
\begin{Highlighting}[]
\NormalTok{\_hi\_mom }\OtherTok{\textless{}{-}} \DecValTok{5}\SpecialCharTok{\^{}}\DecValTok{2}
\end{Highlighting}
\end{Shaded}

\begin{verbatim}
## Error: <text>:1:2: unexpected symbol
## 1: _hi_mom
##      ^
\end{verbatim}
\end{frame}

\begin{frame}[fragile]{Common Errors}
\phantomsection\label{common-errors-1}
The errors saying ``unexpected symbols'' or ``unexpected input'' are R's
way of telling you that these names are not allowed, and that you should
use a different name. Here's how we'd correct these:

\begin{Shaded}
\begin{Highlighting}[]
\NormalTok{tiktok }\OtherTok{\textless{}{-}} \DecValTok{12}

\NormalTok{forever }\OtherTok{\textless{}{-}} \DecValTok{4} \SpecialCharTok{*} \DecValTok{2}

\NormalTok{hi\_mom }\OtherTok{\textless{}{-}} \DecValTok{5}\SpecialCharTok{\^{}}\DecValTok{2}
\end{Highlighting}
\end{Shaded}
\end{frame}

\begin{frame}{Knit Your Document One Final Time}
\phantomsection\label{knit-your-document-one-final-time}
At the end of any of the labs we do, be sure to knit one more time. This
is so you have a finalized version of your work just in case you have
made any changes.

Be sure that a popup window shows you the file! Otherwise there is an
error in the creation of the document. If this happens, you will need to
troubleshoot the R markdown file. I am happy to help you with this.

Also be sure that you read over your file and make sure that it looks
right! If pieces are missing, then you have an error somewhere in the
creation of the document. You should troubleshoot the R Markdown file.
\end{frame}

\begin{frame}{Where Do We Go from Here?}
\phantomsection\label{where-do-we-go-from-here}
During our next section meeting, we will use R for data analysis. You'll
learn how to

\begin{itemize}
\tightlist
\item
  load a data file
\item
  examine the data file
\item
  create tables, graphical displays, and summary statistics
\end{itemize}
\end{frame}

\section{Questions}\label{questions}

\begin{frame}{What Questions Do You Have?}
\phantomsection\label{what-questions-do-you-have}
\end{frame}

\end{document}
